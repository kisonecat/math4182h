\documentclass{homework}
\course{Math 4182H}
\author{Jim Fowler}
\hwtitle{Honors Analysis II}
\usepackage{amsmath}
\DeclareMathOperator{\Mat}{Mat}
\DeclareMathOperator{\End}{End}
\DeclareMathOperator{\Hom}{Hom}
\DeclareMathOperator{\id}{id}
\DeclareMathOperator{\image}{im}
\DeclareMathOperator{\Imag}{Imag}
\DeclareMathOperator{\rank}{rank}
\DeclareMathOperator{\nullity}{nullity}
\DeclareMathOperator{\trace}{tr}
\DeclareMathOperator{\Spec}{Spec}
\DeclareMathOperator{\Sym}{Sym}
\DeclareMathOperator{\pf}{pf}
\DeclareMathOperator{\Ortho}{O}

\newcommand{\R}{\mathbb{R}}
\newcommand{\C}{\mathbb{C}}
\newcommand{\Z}{\mathbb{Z}}
\newcommand{\N}{\mathbb{N}}

\DeclareMathOperator{\sla}{\mathfrak{sl}}
\newcommand{\norm}[1]{\left\lVert#1\right\rVert}
\newcommand{\transpose}{\intercal}


\begin{document}
\maketitle

\begin{inspiration}
\emph{`Tis a derivative from me to mine, and only that I stand for.}\\
\byline{Act 3, Scene 2, Shakespeare's \textit{The Winter's Tale}}
\end{inspiration}

%%%%%%%%%%%%%%%%%%%%%%%%%%%%%%%%%%%%%%%%%%%%%%%%%%%%%%%%%%%%%%%%
\section{Terminology}

\begin{problem}
Let $f:\R^n\to\R^m$ and $a\in\R^n$.  What does ``$f$ is differentiable at $a$'' mean?

Define the \emph{directional derivative} of $f$ at $a$ in the direction of a vector $v\in\R^n$.
\end{problem}

\begin{problem}
Let $f:\R^n\to\R$ be differentiable at $a\in\R^n$. Define the gradient $\nabla f(a)$.
\end{problem}

%%%%%%%%%%%%%%%%%%%%%%%%%%%%%%%%%%%%%%%%%%%%%%%%%%%%%%%%%%%%%%%%
\section{Numericals}

\begin{problem}
Define $F:\R^2\to\R^2$ by $F(x,y)=(x^2-xy,\; x+y^2)$.
Compute $DF(x,y)$ and use it to write the ``best'' linear approximation of $F$ at $(1,2)$.
\end{problem}

\begin{problem}
Let $A$ be an $m\times n$ matrix and $b\in\R^m$.  Define $F:\R^n\to\R^m$ by $F(x)=Ax+b$.
Compute the derivative of $F$.
\end{problem}

\begin{problem}
Let $f,g:\R^n\to\R^m$ be differentiable at $a\in\R^n$.  Define 
\(h:\R^n\to\R\) by \(h(x)=f(x)\cdot g(x)\), i.e., $h(x)$ is defined to be the dot product of $f(x)$ and $g(x)$.

Compute $Dh$ in terms of $f$ and $g$ and $Df$ and $Dg$.
\end{problem}

\begin{problem}
Recall that for
\(
\displaystyle\begin{pmatrix}x&y\\ z&w\end{pmatrix}\)
we have
\(\displaystyle\det\begin{pmatrix}x&y\\ z&w\end{pmatrix}=xw-yz\).

Let
\(
I=\displaystyle\begin{pmatrix}1&0\\0&1\end{pmatrix}\)
and \(H=\displaystyle\begin{pmatrix}a&b\\ c&d\end{pmatrix}\).

Compute the derivative of $\det$ at $I$ in the direction $H$.
\end{problem}

\begin{problem}
Let $\mathrm{inv}:GL_2(\R)\to M_2(\R)$ be the map sending a $2$-by-$2$ matrix to its inverse.

Fix $H\in M_2(\R)$.
Compute the derivative of the inverse map at the identity $I$ in the direction $H$.
\end{problem}

\begin{problem}\label{directional-computation}
Define $g:\R^2\to\R$ by
\[
g(x,y)=
\begin{cases}
\dfrac{x^2y}{x^4+y^2} & (x,y)\neq(0,0), \\
0 & (x,y)=(0,0).
\end{cases}
\]
For $v \in \R^2$, compute the directional derivative $D_v g(0,0)$.
\end{problem}

%%%%%%%%%%%%%%%%%%%%%%%%%%%%%%%%%%%%%%%%%%%%%%%%%%%%%%%%%%%%%%%%
\section{Exploration}

\begin{problem}
Is the function $g$ in \ref{directional-computation} differentiable at $(0,0)$?
\end{problem}

\begin{problem}
Define $f:\R^2\to\R$ by \(f(x,y)=\sqrt{1+x^2+y^2}\).

Show that for all $(x,y) \in \R^2$, \[ 0\le f(x,y)-1 \le \frac12(x^2+y^2). \]

Use this fact to bound the error of a linear approximation to $f$ at $(0,0)$.

Finally find an explicit value of $\epsilon$ so that \(1 - \epsilon < \sqrt{1+0.01^2+0.02^2} < 1 + \epsilon\).
\end{problem}

\begin{problem}
Let $f:\R^n\to\R^m$ be differentiable at $a$.
Prove that $f$ is continuous at $a$.
\end{problem}

\begin{problem}
Let $f,g:\R^n\to\R^m$ be differentiable at $a\in\R^n$.
Prove \emph{from the definition of differentiability} that $f+g$ is differentiable at $a$ and that
\[
D(f+g)(a)=Df(a)+Dg(a).
\]
\end{problem}

\begin{problem}
Let $f:\R^n\to\R$ be differentiable at $a$ with gradient $\nabla f(a)\neq 0$.

Prove that among all unit vectors $u$, the maximum of $D_u f(a)$ equals $\|\nabla f(a)\|$.

Then show that this maximum occurs in the direction $u=\nabla f(a)/\|\nabla f(a)\|$.
\end{problem}


\begin{problem}
Let $f:\C\to\C$ and write $z=x+iy$.  Suppose
\[
f(z)=u(x,y)+iv(x,y)
\]
with $u,v:\R^2\to\R$.

What does it mean to say that $f$ is \emph{complex differentiable} at $z_0=x_0+iy_0$?

When $f$ is complex differentiable, verify that the \textbf{Cauchy--Riemann equations} hold at $(x_0,y_0)$, i.e., show that
\[
u_x(x_0,y_0)=v_y(x_0,y_0),\qquad u_y(x_0,y_0)=-v_x(x_0,y_0).
\]
\end{problem}

%%%%%%%%%%%%%%%%%%%%%%%%%%%%%%%%%%%%%%%%%%%%%%%%%%%%%%%%%%%%%%%%
\section{Prove or Disprove and Salvage if Possible}

{\footnotesize For the following problems, first decide whether the statement is true or false as written. If it is true, give a complete proof that works in full generality (not just examples!) and make clear where the hypotheses are used. If it is false, disprove it with a counterexample: give a specific example that satisfies the assumptions but violates the conclusion. Justify that your example is actually a counterexample! Then try to salvage the statement by making a minimal, natural change---that could be adding a missing hypothesis or weakening the conclusion. State your corrected version clearly and prove it.}

\begin{problem}
If all partial derivatives of $f:\R^2\to\R$ exist at $a$, then $f$ is differentiable at $a$.
\end{problem}

\begin{problem}
If for $f:\R^n\to\R$ all directional derivatives $D_u f(a)$ exist when $\|u\|=1$, and the map $u\mapsto D_u f(a)$ is continuous on the unit sphere, then $f$ is differentiable at $a$.
\end{problem}

\end{document}
