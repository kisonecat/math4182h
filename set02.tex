\documentclass{homework}
\course{Math 4182H}
\author{Jim Fowler}
\hwtitle{Honors Analysis II}
\usepackage{amsmath}
\DeclareMathOperator{\Mat}{Mat}
\DeclareMathOperator{\End}{End}
\DeclareMathOperator{\Hom}{Hom}
\DeclareMathOperator{\id}{id}
\DeclareMathOperator{\image}{im}
\DeclareMathOperator{\Imag}{Imag}
\DeclareMathOperator{\rank}{rank}
\DeclareMathOperator{\nullity}{nullity}
\DeclareMathOperator{\trace}{tr}
\DeclareMathOperator{\Spec}{Spec}
\DeclareMathOperator{\Sym}{Sym}
\DeclareMathOperator{\pf}{pf}
\DeclareMathOperator{\Ortho}{O}

\newcommand{\R}{\mathbb{R}}
\newcommand{\C}{\mathbb{C}}
\newcommand{\Z}{\mathbb{Z}}
\newcommand{\N}{\mathbb{N}}

\DeclareMathOperator{\sla}{\mathfrak{sl}}
\newcommand{\norm}[1]{\left\lVert#1\right\rVert}
\newcommand{\transpose}{\intercal}


\begin{document}
\maketitle

\begin{inspiration}
  \textit{Omnia mutantur, nihil interit.} \\
  \byline{Ovid's \textit{Metamorphoses}, perhaps not about derivatives vanishing.}
\end{inspiration}

\section{Terminology}

\begin{problem}
  Write down a statement of the \textbf{chain rule} for multivariable functions.
\end{problem}

\begin{problem}
  What is meant by the \textbf{G\^ateaux derivative} of a function $f : \R^n \to \R^m$?

  Compare this to \ref{gateaux-derivative-podasip}.
\end{problem}

\begin{problem}
  What is meant by the \textbf{Fr\'echet derivative} of a function $f : \R^n \to \R^m$?
\end{problem}


\begin{problem}\label{tangent-space}%
  For a smooth function $f : \R^3 \to \R$, consider the level set $f^{-1}(0) = \{ (x,y,z) \in \R^3 : f(x,y,z) = 0 \}$. For a point $(x,y,z) \in f^{-1}(0)$, define the \textbf{tangent space} to $f^{-1}(0)$ at the point $(x,y,z)$, denoted $T_{(x,y,z)} f^{-1}(0)$.
\end{problem}

\section{Numericals}

\begin{problem}
Let $g:\R^2\to\R$ be $C^1$ with $\nabla g(2,-1)=(4,7)$.
Define $f:\R^2\to\R^2$ via
\[
f(x,y)=(x+y,\;x-2y).
\]
Then compute $\nabla(g\circ f)(1,1)$.
\end{problem}

\begin{problem}
  Consider a ``change of variables'' where we relate $(x,y)$ to $(u,v)$ via
  \[
    x = u + v, \qquad
    y = v.
  \]
  Verify that $\displaystyle\frac{\partial y}{\partial u} = 0$ and yet $\displaystyle\frac{\partial u}{\partial y} \neq 0$.

This is the so-called \textbf{second fundamental confusion of calculus}, cf. \textsection 10.3 of Penrose's \textit{The Road to Reality}.
\end{problem}

\begin{problem}\label{laplacian-in-polar}%
  Use the chain rule to rewrite the Laplacian $\Delta = \displaystyle\frac{\partial^2}{\partial x^2} + \displaystyle\frac{\partial^2}{\partial y^2}$ in terms of polar coordinates, i.e., in terms of $\displaystyle\frac{\partial}{\partial r}$ and $\displaystyle\frac{\partial}{\partial \theta}$.
\end{problem}

\section{Exploration}

\begin{problem}\label{riesz-representation}%
  Suppose $L : \R^n \to \R$ is linear. Show that there exists $\vec{v} \in \R^n$ so that $L(\vec{a}) = \vec{v} \cdot \vec{a}$.

Use this to explain \ref{gradient-definition}.
\end{problem}

\begin{problem}
  Recall \ref{cauchy-riemann-equations}. Suppose $u, v : \R^2 \to \R$ are smooth functions satisfying
\[
u_x(x,y)=v_y(x,y)\qquad \text{and} \qquad u_y(x,y)=-v_x(x,y),
\]
for all $(x,y) \in \R^2$. Show that $\Delta u = 0$ and $\Delta v = 0$.
\end{problem}

\begin{problem}
  Let $\Phi:\R^2\to\R^2$ be a rotation by $\theta$ and a translation by $(a,b)$, i.e.,
  \[
    \Phi(x,y)=\bigl(x\cos\theta - y\sin\theta + a,\; x\sin\theta + y\cos\theta + b\bigr).
  \]
If $u$ is $C^2$ and $\Delta u=0$, show that $\Delta(u\circ \Phi) = 0$.

Find some other map $\Psi : \R^2 \to \R^2$ which isn't just a rotation or a translation but with the property that $\Delta u = 0$ implies $\Delta (u \circ \Psi) = 0$.
\end{problem}

\begin{problem}
  Suppose $f : \R^n \to \R$ is continuously differentiable and $f$ is \textbf{positively homogeneous of degree $\mathbf{k}$}, meaning $f(\lambda \cdot \vec{v}) = \lambda^k f(\vec{v})$ for all $\vec{v} \in \R^n$ and $\lambda > 0$. Prove
  \[
    k\,f(x_{1},\ldots ,x_{n})=\sum _{i=1}^{n}x_{i}{\frac {\partial f}{\partial x_{i}}}(x_{1},\ldots ,x_{n}).
  \]
  This is the first half of \textbf{Euler's homogeneous function theorem}.
\end{problem}

\begin{problem}
  For a positive integer $k$, consider the function $f_k : \R^2 \to \R$ defined in polar coordinates via $f_k(r,\theta) = r^k \cos(k\theta)$.

  Note that $f_k$ is homogeneous of degree $k$, and 
  use the formula in \ref{laplacian-in-polar} to check $\Delta f_k = 0$.

  Then write $f_k$ in Cartesian coordinates to find some \textbf{harmonic} polynomials of degree $k$.
\end{problem}

\begin{problem}
Suppose $f : \R^3 \to \R$ is a smooth function, and $f(0,0,0) = 0$ and $\nabla f(0,0,0) \neq 0$. Explain why for $v \in T_{(0,0,0)} f^{-1}(0)$ we have $v \cdot \nabla f(0,0,0) = 0$.
\end{problem}

\section{Prove or Disprove and Salvage if Possible}

\begin{problem}
  If all directional derivatives vanish at $a$, then the function is differentiable at $a$.
\end{problem}

\begin{problem}
  Let $\gamma:\R\to\R^2$ be a $C^2$ curve parametrized by arclength, meaning $|\gamma'(s)| \equiv 1$. Then $\gamma'(s)\cdot \gamma''(s)=0$. \textit{Hint:} use \ref{dot-product-derivative}.
\end{problem}

\begin{problem}
  Let $f:\R^n\to\R^m$ be differentiable at $a$, and let $g:\R^m\to\R$ be differentiable at $f(a)$.
  Define $\phi(t)=g\bigl(f(a+t v)\bigr)$ for $v\in\R^n$.

  Then
\[
\nabla(g\circ f)(a)=Df(a)\left( \left(\nabla g\right)(f(a)) \right).
\]
% Missing transpose!
\end{problem}

\begin{problem}
  If $f : \R^n \to \R^m$ is differentiable at $a$ and $g : \R^m \to \R$ has all directional derivatives at $f(a)$, then
  \[
    D_v (g \circ f)(a) = (D_{Df(a)(v)} g)(f(a)).
  \]
\end{problem}

\end{document}
